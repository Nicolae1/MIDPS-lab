
\section*{Concluzie}


Dezvoltarea web este un termen larg, care cuprinde orice activitate legata de dezvoltarea unui sit web. Aceasta poate include dezvoltarea afacerilor prin comert electronic (e-commerce), web design, dezvoltarea de continut web, programare specifica, configurarea serverelor web, etc. Dezvoltarea web include atit realizarea unor simple pagini web statice cu text, până la cele mai complexe aplicaţii Internet, afaceri electronice (ebusiness), sau servicii de reţele sociale.


In aceasta lucrare de laborator mi-am dezvoltat abilitatile practice in Web Development. Am creat primul meu website, lucru care la prima vedere imi paruse complicat. Am intilnit probleme dar cu calm, documentindu-ma ale-am rezolvat. Cunoaterea web development este foarte importanta si interesanta ca student IT.


Proiectarea paginilor web este un proces de conceptualizare, planificare, modelare şi execuţie a conţinutului media electronic livrat pe Internet într-o formă tehnică adecvată pentru interpretarea şi afişarea într-un browser web sau altă interfaţă grafică pentru utilizatori (graphical user interface, GUI).

Scopul proiectării web este crearea unui sit web (o colecţie de fişiere electronice încărcate pe unul sau mai multe servere) cu conţinut (care poate include caracteristici sau interfeţe interactive) pentru utilizatorul final sub forma paginilor web.
\clearpage

      
